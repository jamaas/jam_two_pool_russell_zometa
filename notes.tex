% Created 2026-01-30 Fri 14:23
% Intended LaTeX compiler: pdflatex
\documentclass[11pt]{article}
\usepackage[utf8]{inputenc}
\usepackage[T1]{fontenc}
\usepackage{graphicx}
\usepackage{longtable}
\usepackage{wrapfig}
\usepackage{rotating}
\usepackage[normalem]{ulem}
\usepackage{amsmath}
\usepackage{amssymb}
\usepackage{capt-of}
\usepackage{hyperref}
\author{Jim Maas}
\date{\today}
\title{}
\hypersetup{
 pdfauthor={Jim Maas},
 pdftitle={},
 pdfkeywords={},
 pdfsubject={},
 pdfcreator={Emacs 30.1 (Org mode 9.7.11)}, 
 pdflang={English}}
\begin{document}

\tableofcontents

\textit{{[}2026-01-25 Sun 09:54]}attempting to figure out why lines 45-45 seem to
be pointing to wrong elements of I vector?

// Initial state. State values of  QA0, QB0, QT0\\
let y0 = State::new(I[2], I[3], I[4]);

think if should be I 3, 4, 5?
\textit{{[}2026-01-25 Sun 11:50]}C-u C-c !

going create a new model script, strip out the "total pool" stuff and
see if I can graph the concentrations

new name will be "jam\textsubscript{two}\textsubscript{pool}\textsubscript{no}\textsubscript{total}"

remember in Rust , vector designations begin at 0 \ldots{}

\textit{{[}2026-01-26 Mon 12:39]}sent text to Sylvain Renevey asking for help,
also put a question on Rust users group asking for suggestions

\begin{itemize}
\item need to ask Sylvain why both
```// declare the external Rust crates required
//use ode\textsubscript{solvers}::rk4::*;
use ode\textsubscript{solvers}::*;```

\begin{itemize}
\item should be every feature, therefor rk4 line not necessary, covered by others?
\end{itemize}
\end{itemize}
\textit{{[}2026-01-27 Tue 10:41] } Professor Juan Zometa from the International
University of Berlin, answered my question on Rust Users group, and
said that he had encountered the same problem using the ode\textsubscript{solvers}
crate, inability to extract non-state variables!  He solved his
problem by changing ode solvers written as part of the Russell
consortium called russell\textsubscript{ode}.  Then he converted a model he had of a
mechanical system to my bio one, using the russell\textsubscript{ode} solver and it
appears to work great!  Now to test it and try it, next to see if I
can graph!

\textit{{[}2026-01-28 Wed 12:16] } Very nice code from Pablo Zometa, works great,
hope I can ask him a few questions to learn more about rust
\textit{{[}2026-01-30 Fri 14:18] } Can output data directly from rust using > to
print directly to file Then start gnuplot, then "plot 'results.tx'
will show the plot!  Assumes first column is x, second column is y,
can also do multiple colums, some instructions at
\url{https://alvinalexander.com/technology/gnuplot-charts-graphs-examples/}
likely good enough to start with
\end{document}
